%darc, the Durham Adaptive optics Real-time Controller.
%Copyright (C) 2010 Alastair Basden.

%This program is free software: you can redistribute it and/or modify
%it under the terms of the GNU Affero General Public License as
%published by the Free Software Foundation, either version 3 of the
%License, or (at your option) any later version.

%This program is distributed in the hope that it will be useful,
%but WITHOUT ANY WARRANTY; without even the implied warranty of
%MERCHANTABILITY or FITNESS FOR A PARTICULAR PURPOSE.  See the
%GNU Affero General Public License for more details.

%You should have received a copy of the GNU Affero General Public License
%along with this program.  If not, see <http://www.gnu.org/licenses/>.

\begin{verbatim}
QUICKSTART for Paris, Oct 2009

To get the system into a working 2 camera mode:

Plug monitor and keyboard into telemetry server (small HP machine)
Network cable (motherboard) to your external network.
Network cable (PCI) to the local switch.
You will need to configure this machine for external network access.


Boot the telemetry server (small HP machine).
Wait until it has booted (5 minutes to allow crontab time to run).
It logs on automatically.
Open a terminal.
cd /Canary/rtc
./recvStream.py


Plug Network cable on camera PC (black dell machine) into the switch.
Plug the camera fibre into the right hand port.
Plug the camera trigger for the 2 cameras into the cameras and into
the two right hand BNC connectors.
Boot the camera PC (black Dell machine)

Plug network cable on RTC PC (mac pro) into port 1, and into the switch.
Plug the camera fibre into port 0 (right hand port).

Boot the RTC (mac pro).
Log into the RTC as canary twice (ssh from the telemetry server: ssh -X rtc ).
In each, type:
source /Canary/etc/rtc.bashrc
In one type:
cd rtc
./control.py

In the other type:
cd rtc
./startStreams.py -h192.168.0.1

Note - eventually, this should be self-starting, but when we changed
the network config just before shipping it stopped working, so please
do it manually for now.


On the telemetry server, 
open a terminal and type:
cd /Canary/rtc
./rtcgui.py


Make sure that you have a suitable config file (e.g. config2camera.py
or configshakti.py) in /Canary/rtc/ on the telemetry server and in
/home/canary/rtc/ on the rtc machine (mac pro).  

This is currently a bug, which will be fixed (we didn't want to fix it
in the hour before shipping!).

On the GUI, click the start button.  Choose the config file
config2camera.py.  The RTC should then start running.

If working in 1 camera mode, use configshakti.py.

Go to the streams tab, and turn on some streams.  
For example put 100 100 100 in the 3 entry boxes next to rtcPxlBuf.
This sets the decimate rates (for this GUI, out of the RTC between RTC
and tememetry server).
Do this for rtcPxlBuf

Click the spawn plot button.
Select the rtcPxlBuf button in the subscribe too... window, then close
the subscribe too... window.

This will then start showing camera pixels.
To view as an image, type into the text entry box in the bottom left
of the plot:
data.shape=128,256
and then click elsewhere on the screen (so that
the text entry looses focus).
This will show interleaved camera data.
To view from just one camera do either:
data.shape=128,256
data=data[:,::2]
or
data.shape=128,256
data=data[:,1::2]

Any valid python code can be inserted here.  It can be used for
implementing overlays or anything else you wish.

Note, to avoid having to do this, you can set up some plots, and save
them to a configuration file.  Some examples have been done for you -
run the GUI with:
./rtcgui.py initPhaseA.py
A list of available plots then appears in the GUI.
(some of these will work with 2 camera mode, some with 1 camera mode).
These can then be toggled to display or remove plots.

Feel free to edit initPhaseA.py as you wish.

To run some test scripts (some examples of how to do things with the camera):
cd /Canary/local/scripts/test
and look at these scripts.

Note - if any of these scripts require rtc output (pixels/slopes etc),
then the streams need to be turned on in the GUI first, as the script
facility to turn streams on hasn't been implemented yet.


\end{verbatim}
